%%%%%%%%%%%%%%%%%%%%%%%%%%%%%%%%%%%%%%%%%%%%%%%%%%%%%%%%%%%%%%%%%%%%%%%%%%%%%%%%

\pagestyle{empty}
\begin{abstract}

Unha tarefa pouco apreciada e minimizada do traballo dun técnico consiste no control administrativo dos activos que xestiona. Este control é imprescindible para unha xestión colectiva dos recursos das empresas, de xeito que poidan expandirse máis aló da capacidade dunha persoa. Porén, non é sinxelo transformar un sistema manual e improvisado, cun longo historial e no que dependen moitos procesos vitais para as empresas, nun sistema automatizado e axeitado para o entorno empresarial multitudinario. O obxectivo deste proxecto é a creación dunha plataforma para a xestión de activos informáticos que lle permita ao persoal técnico axilizar o seu traballo de xestionar e controlar os activos existentes dentro da organización.

\vspace*{25pt}
\begin{segundoresumo}
	
% !TeX spellcheck = gl_ES

Administrative control of assets is an often overlooked and underappreciated part of an administrator's job. This is a must for any company that wants to expand its management outside the capabilities of a single person. However, it's hard to transform an ad-hoc process, with a long history and on which many vital processes depend on, to an automatic system fit for big plural enterprise environments. The objective of this project is the creation of a platform for the management of computer assets that helps technical personnel in managing and controlling existing assets inside an organization.

\end{segundoresumo}

\vspace*{25pt}
\begin{multicols}{2}
\begin{description}
\item [\palabraschaveprincipal:] \mbox{} \\[-20pt]
    \item Xestión
    \item Sistemas
    \item Java
    \item Web
    \item REST
    \item Thymeleaf
    \item Spring
\end{description}
\begin{description}
\item [\palabraschavesecundaria:] \mbox{} \\[-20pt]
    \item Management
    \item Systems
    \item Java
    \item Web
    \item REST
    \item Thymeleaf
    \item Spring
\end{description}
\end{multicols}

\newpage
\vspace*{\fill}
Esta memoria, incluíndo todas as súas imaxes, atópase baixo a licencia Creative Commons Atribución-Compartir do mesmo xeito (CC BY-SA 4.0)\footnote{\url{https://creativecommons.org/licenses/by-sa/4.0/}}
\end{abstract}
\pagestyle{fancy}

%%%%%%%%%%%%%%%%%%%%%%%%%%%%%%%%%%%%%%%%%%%%%%%%%%%%%%%%%%%%%%%%%%%%%%%%%%%%%%%%
