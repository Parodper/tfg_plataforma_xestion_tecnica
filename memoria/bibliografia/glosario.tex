%%%%%%%%%%%%%%%%%%%%%%%%%%%%%%%%%%%%%%%%%%%%%%%%%%%%%%%%%%%%%%%%%%%%%%%%%%%%%%%%
% Obxectivo: Lista de termos empregados no documento,                          %
%            xunto cos seus respectivos significados.                          %
%%%%%%%%%%%%%%%%%%%%%%%%%%%%%%%%%%%%%%%%%%%%%%%%%%%%%%%%%%%%%%%%%%%%%%%%%%%%%%%%

\newglossaryentry{ITAMg}{
  name=ITAM,
  description={Ferramenta de xestión de activos, que axuda ao persoal técnico a controlar os distintos activos dunha empresa}
}

\newglossaryentry{Template}{
  name=Template,
  description={Modelo a partir do cal sácanse \Gls{Component} (véxase tamén a sección \ref{obx:template})}
}

\newglossaryentry{Component}{
  name=Component,
  description={Compoñente xerado a partir dun \Gls{Template} (véxase tamén a sección \ref{obx:component})}
}

\newglossaryentry{TemplateField}{
  name=TemplateField,
  description={Campo dun \Gls{Template}, a partir do cal sácanse \Gls{Field} (véxase tamén a sección \ref{obx:tfield})}
}

\newglossaryentry{Field}{
  name=Field,
  description={Campo dun \Gls{Component}, creado a partir dun \Gls{TemplateField} (véxase tamén a sección \ref{obx:field})}
}

\newglossaryentry{User}{
  name=User,
  description={Clase que representa un usuario da aplicación (véxase tamén a sección \ref{obx:user})}
}

\newglossaryentry{BCrypt}{
  name=BCrypt,
  description={Algoritmo de \Gls{hash} para contrasinais, que inclúe unha sal}
}

\newglossaryentry{hash}{
  name=Suma de verificación,
  description={A suma de verificación, ou resumo criptográfico,\cite{digatic} (coñecida en inglés coma «\textit{hash}»), é unha función matemática sobrexectiva que permite verificar a igualdade de dous datos, sen precisar dunha comparación exhaustiva}
}

\newglossaryentry{testemuna}{
  name=testemuña,
  plural=testemuñas,
  description={Unha testemuña (en inglés «\textit{token}»), é un valor que serve como proba ou identificación do seu portador fronte un servizo}
}

\newglossaryentry{SPAg}{
  name=SPA,
  description={Clase de aplicación web, deseñada de xeito que non precise recargarse a páxina, usando JavaScript para facer transicións só dende o cliente}
}

\newglossaryentry{openapi}{
  name=OpenAPI,
  description={Normativa que define unha sintaxe para describir interfaces REST\cite{openapi}}
}