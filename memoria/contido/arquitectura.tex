\chapter{Arquitectura}
\label{chap:arquitectura}

A plataforma está formada por dous compoñentes: unha parte servidor, implementada en Java e que realiza as labores de conexión coa base de datos, e que presenta unha \acrshort{API} \acrshort{REST} ao exterior; e unha parte cliente, un servidor Web en Java que conecta á parte servidor mediante peticións á \acrshort{API} \acrshort{REST}. Decidiuse engadir unha API REST, e non só integrar todo no mesmo sistema, para permitir futuras extensións e aplicacións alternativas, que permita un acceso máis cómodo e adaptado ao persoal técnico. O esquema de deseño fundamental foi «\textit{API first}», pensando en presentar un mesmo método de interacción mediante peticións HTTP con JSON, no canto de integrar o cliente dentro da aplicación e que chamar directamente aos métodos dende Java.

\section{Obxectos}

Existen cinco obxectos principais. A figura \ref{fig:diagrama_clases} representa as relacións entre as clases:

\subsection{Template}
\label{obx:template}

Representa un modelo a partir do cal sacar \Gls{Component}.

\textbf{Campos:}

\begin{description}
    \item[id] Valor numérico que representa o ID do modelo
    \item[name] Cadea que contén o nome do modelo
    \item[description] Cadea que conten unha pequena descrición do modelo
    \item[fields] Lista de clases \Gls{TemplateField}, que son os campos do modelo
\end{description}

\subsection{Component}
\label{obx:component}

Representa un compoñente creado a partir dun \Gls{Template}.

\textbf{Campos:}

\begin{description}
    \item[id] Valor numérico que representa o ID do compoñente
    \item[name] Cadea que contén o nome do compoñente
    \item[description] Cadea que conten unha pequena descrición do compoñente
    \item[template] Clave foránea/punteiro que apunta ao \Gls{Template} do que se creou este Component
    \item[fields] Lista de clases \Gls{Field}, que son os campos do compoñente
\end{description}

\subsection{TemplateField}
\label{obx:tfield}

Representa os campos do modelo, que son a base para crear os campos do \Gls{Component}.

\textbf{Campos:}

\begin{description}
    \item[id] Valor numérico que representa o ID do campo
    \item[name] Cadea que contén o nome do campo. Non se poden repetir no mesmo \Gls{Template}
    \item[mandatory] Valor binario que indica se o campo é obrigatorio
    \item[type] Indica o tipo do campo con un enumerado.
\end{description}

Os tipos de campos son:

\begin{description}
    \item[\texttt{TEXT}] Permite gardar texto plano e sen formato
    \item[\texttt{LINK}] Representa unha ligazón a outro compoñente
    \item[\texttt{DATETIME}] Representa unha data
\end{description}

\subsection{Field}
\label{obx:field}

Representa un campo dun \Gls{Component}.

\textbf{Campos:}

\begin{description}
    \item[id] Valor numérico que representa o ID do campo
    \item[type] Tipo do campo, copiado do \Gls{TemplateField} e usado para diferenciar durante a serialización
    \item[content] Atributo «virtual», que serve para indicarlle a OpenAPI que clases poden conterse dentro do campo.
    \item[templateField] Ligazón/clave foránea no modelo relacional ao campo do \Gls{Template} no que se basea este campo o \Gls{Component} 
\end{description}

O obxecto \texttt{Field} é unha clase abstracta, e parametrizada cunha clase xenérica que serve para devolver distintos tipos de obxectos. Para poder usar herdanza na base de datos hai tres tipos de estratexias:\cite{herdanza}

\begin{itemize}
	\item \texttt{MappedSuperclass}, que crea unha táboa por cada subclase (pero non para a superclase) replicando todas as columnas herdadas e propias. O problema que ten isto é que non se pode referenciar á superclase dende outras táboas da base de datos, o que neste caso impediría que \texttt{Component} tivera unha foránea a \texttt{Field}, e tería que ter foráneas a todos os campos individuais.
	\item Reunir nunha soa táboa todas as subclases. Isto permite engadir foráneas, pero o número de subclases e o rendemento están limitados polo número de columnas que acepte a base de datos.
	\item Unha táboa por subclase, de xeito que para acceder a unha entidade precísase facer unha operación de \texttt{JOIN} entre a táboa da subclase e da superclase. Non ocupa tanto espazo coma a seguinte opción, pero é máis lenta ca a anterior opción por mor de precisar acceder a varias táboas.
	\item Unha táboa completa por subclase, tendo cada táboa da subclase os seus atributos máis os da superclase. Non se precisan facer \texttt{JOIN}, pero ocupa máis espazo ao ter que duplicar todos os atributos.
\end{itemize}

Escolleuse a 2ª opción por ser a máis sinxela, eficiente e ser a opción predeterminada en Jakarta. Ademais, non se considera que o número de tipos de campos vaia medrar dabondo coma para ser un problema. E, en calquera caso, poderíase migrar a calquera das outras opcións só cambiando unha anotación.

As subclases existentes son:

\begin{description}
    \item[\texttt{TextField}] Representa un campo de texto. Substitúe a clase xenérica por unha clase \texttt{String}
    \item[\texttt{DatetimeField}] Representa un campo de texto. Substitúe a clase xenérica por unha clase \texttt{JSONDatetime}
    \item[\texttt{LinkField}] Representa unha ligazón a outro compoñente.  Substitúe a clase xenérica por unha clave foránea a unha clase \texttt{Component}
    \item[\texttt{NumberField}] Representa un valor numérico decimal. Substitúe a clase xenérica por \texttt{BigDecimal}
\end{description}

Inda que cada subclase ten unha propiedade \texttt{content} distinta para evitar colisións na base de datos, o contido é o mesmo.

\subsection{User}
\label{obx:user}

Representa un usuario no sistema.

\textbf{Campos:}

\begin{description}
    \item[id] Valor numérico que representa o ID do usuario
    \item[username] Cadea que contén o alcume do usuario
    \item[email] Cadea que contén o enderezo electrónico do usuario
    \item[password] Cadea que contén o contrasinal do usuario, cifrado con \Gls{BCrypt}
    \item[role] Enumerado que contén o tipo de rol do usuario
\end{description}

Os posibles valores para o campo role son \texttt{ADMINISTRATOR} e \texttt{NORMAL\_USER}. Para máis información, consulte a sección \ref{sec:usuarios}.

Para os usuarios tamén existe a clase Token, que, aparte do usuario ao que referencia e un ID, garda unha \Gls{testemuna} nunha cadea.

\begin{figure}\hspace*{-3cm}
    \centering
    \resizebox{.9\textwidth}{!}{\begin{tikzpicture}
\begin{class}{Template}{20,0}
	\attribute{id : long}
	\attribute{name : String}
	\attribute{description : String}
	\attribute{fields : TemplateField[0..*]}
\end{class}
\begin{class}{TemplateField}{10,0}
	\attribute{id : long}
	\attribute{name : String}
	\attribute{mandatory : boolean}
	\attribute{type : FieldTypes}
\end{class}
\begin{class}{FieldTypes}{0,0}
	\attribute{TEXT}
	\attribute{DATETIME}
	\attribute{LINK}
	\attribute{NUMBER}
\end{class}

\composition{Template}{0..*}{fields}{TemplateField}
\aggregation{TemplateField}{1}{type}{FieldTypes}

\begin{class}{Component}{17,5}
	\attribute{id : long}
	\attribute{name : String}
	\attribute{description : String}
	\attribute{fields : Field[0..*]}
	\attribute{template : Template}
\end{class}
\begin{abstractclass}[template=T]{Field}{8,7}
	\attribute{id : long}
	\attribute{type : FieldTypes}
	\attribute{templateField : TemplateFields}
	\attribute{content : T}
	\operation{getContent() : T}
	\operation{setContent(content : T) }
\end{abstractclass}
\begin{class}{NumberField}{0,10}
	\inherit{Field}
	\attribute{number\_content : BigDecimal}
	\operation{getContent() : BigDecimal}
	\operation{setContent(content : BigDecimal)}
\end{class}
\begin{class}{TextField}{8,10}
	\inherit{Field}
	\attribute{text\_content : String}
	\operation{getContent() : String}
	\operation{setContent(content : String)}
\end{class}
\begin{class}{DatetimeField}{0,6}
	\inherit{Field}
	\attribute{datetime\_content : JSONDatetime}
	\operation{getContent() : JSONDatetime}
	\operation{setContent(content : JSONDatetime)}
\end{class}
\begin{class}{LinkField}{15,10}
	\inherit{Field}
	\attribute{link\_content : Component}
	\operation{getContent() : Component}
	\operation{setContent(content : Component)}
\end{class}

\composition{Component}{0..*}{fields}{Field}
\unidirectionalAssociation{LinkField}{1}{link\_content}{Component}
\aggregation{Field}{1}{type}{FieldTypes}
\aggregation{Component}{1}{template}{Template}
\aggregation{Field}{1}{templateField}{TemplateField}

\begin{class}{User}{7,15}
	\attribute{id : long}
	\attribute{username : String}
	\attribute{email : String}
	\attribute{password : String}
	\attribute{role : Roles}
\end{class}
\begin{class}{Token}{0,15}
	\attribute{id : long}
	\attribute{user : User}
	\attribute{token : String}
\end{class}
\begin{class}{Roles}{15,15}
	\attribute{NORMAL\_USER}
	\attribute{ADMINISTRATOR}
\end{class}


\aggregation{User}{1}{role}{Roles}
\aggregation{Token}{1}{user}{User}

\end{tikzpicture}}
    \caption{Diagrama de clases do servidor}
    \label{fig:diagrama_clases}
\end{figure}

\section{API REST do servidor}

Todas as rutas que inclúen algún id ou nome, devolven un «\textit{HTTP 404 Not Found}» se non existe algún dos identificadores ou nomes. A API acepta tanto JSON coma XML, pero o cliente só usa JSON.

A meirande parte das peticións precisan dunha testemuña válida pasa na cabeceira «\texttt{X-API-KEY}». Para máis información sobre permisos consultar a sección 
\ref{sec:seguridade}

Todas as rutas descritas son relativas á ruta \texttt{\$SERVIDOR/api/v0/\$SUBSECCIÓN}.

\newpage

\subsection{\texttt{/templates}}

Nesta sección xestiónanse os modelos os seus campos.

\texttt{/}

\begin{description}
    \item[GET] Devolve unha listaxe de todos os modelos dispoñibles. Permite devolver só un rango engadindo os parámetros \texttt{skip} e \texttt{count}
    \item[POST] Crea un novo modelo incluído no corpo da petición. Devolve un «\textit{HTTP 409 CONFLICT}» se xa existe outro modelo co mesmo nome
\end{description}

\texttt{/find?name=(nome)}

\begin{description}
    \item[GET] Devolve unha listaxe de todos os modelos dispoñibles cuxo nome coincide con «nome»
\end{description}

\texttt{/(id)}

\begin{description}
    \item[GET] Devolve o modelo con id \texttt{id}
    \item[POST] Actualiza o modelo co contido do corpo da petición
    \item[DELETE] Elimina o modelo. Devolve un HTTP 409 Bad Request se inda quedan \Gls{Component}s vinculados a este modelo
\end{description}

\texttt{/(id)/components}

\begin{description}
    \item[GET] Devolve unha listaxe de todos os compoñentes derivados deste modelo
\end{description}

\texttt{/(id)/fields}

\begin{description}
    \item[GET] Devolve unha listaxe de todos os campos deste modelo
    \item[POST] Engade un novo campo ao compoñente. Falla se o modelo está en uso
\end{description}

\texttt{/(id)/fields/(id)}

\begin{description}
    \item[GET] Devolve o campo indicado
    \item[POST] Modifica o campo indicado, co contido do corpo da petición
    \item[DELETE] Elimina o campo indicado.
\end{description}

Tanto a modificación coma o borrado fallan se o modelo está en uso.

\subsection{\texttt{/components}}

Nesta sección xestiónanse os compoñentes e os seus campos.

\texttt{/}

\begin{description}
    \item[GET] Devolve unha listaxe de todos os compoñentes dispoñibles. Permite devolver só un rango engadindo os parámetros \texttt{skip} e \texttt{count}
    \item[POST] Crea un novo compoñente incluído no corpo da petición
\end{description}

\texttt{/find?name=(nome)}

\begin{description}
    \item[GET] Devolve unha listaxe de todos os compoñentes dispoñibles cuxo nome coincide con «nome»
\end{description}

\texttt{/(id)}

\begin{description}
    \item[GET] Devolve o compoñente con id \texttt{id}
    \item[POST] Actualiza o compoñente co contido do corpo da petición
    \item[DELETE] Elimina o compoñente
\end{description}

\newpage

\texttt{/(id)/fields}

\begin{description}
    \item[GET] Devolve unha listaxe de todos os campos deste modelo
    \item[POST] Engade un novo campo ao compoñente. Falla se o modelo está en uso
\end{description}

\texttt{/(id)/fields/(nome)}

\begin{description}
    \item[GET] Devolve o campo chamado «nome»
    \item[PUT] Modifica o valor campo indicado, co contido do corpo da petición. A columna modificada na base de datos depende do tipo do dato
    \item[DELETE] Elimina o campo indicado
\end{description}

Se se intenta cambiar un campo obrigatorio a unha cadea baleira ou nula, ou eliminalo, devólvese unha mensaxe «\textit{HTTP 400 Bad Request}».

\subsection{\texttt{/users}}

Nesta sección xestiónanse os usuarios. Os usuarios só os poden modificar e crear usuarios administradores.

\texttt{/}

\begin{description}
    \item[GET] Devolve unha listaxe de todos os usuarios: só se inclúe o ID por seguridade. Permite devolver só un rango engadindo os parámetros \texttt{skip} e \texttt{count}.
    \item[POST] Crea un novo usuario.
\end{description}

\texttt{/login?user=(nome)}

\begin{description}
    \item[POST] Inicia sesión para o usuario «nome». O método precisa que se pase o contrasinal sen cifrar no corpo da mensaxe, e devolve unha \Gls{testemuna}
\end{description}

\texttt{/find?name=(nome)}

\begin{description}
    \item[GET] Devolve o ID do usuario co nome indicado
\end{description}

\texttt{/(id)}

\begin{description}
    \item[GET] Devolve o usuario con id \texttt{id}
    \item[POST] Actualiza o usuario co contido do corpo da petición
    \item[DELETE] Elimina o usuario.
\end{description}

\texttt{/(id)/role}

\begin{description}
    \item[GET] Devolve o rol do usuario
    \item[PUT] Cambia o rol do usuario ao contido do corpo da petición
\end{description}

\texttt{/(id)/logout}

\begin{description}
    \item[POST] Pecha a sesión do usuario, eliminando a \Gls{testemuna} indicado no corpo da petición. Devolve un código «\textit{HTTP 404 Not Found}» se o usuario non ten rexistrada esa testemuña
\end{description}

Se se intenta cambiar un campo obrigatorio a unha cadea baleira ou nula, ou eliminalo, devólvese unha mensaxe «\textit{HTTP 400 Bad Request}».

\subsection{\texttt{/search}}

Baixo esta ruta existen 3 subrutas (\texttt{components}, \texttt{templates} e \texttt{users}). Todas elas funcionan do mesmo xeito: Envíaselle unha petición GET, e devolve unha listaxe dos obxectos axeitados. Coa petición GET pódense incluír os seguintes atributos:

\begin{description}
    \item[query] Cadea que contén a petición a buscar. Véxase a sección \ref{sec:busca}
    \item[skip] Número de elementos que saltar ao inicio da listaxe
    \item[count] Número de elementos a devolver
\end{description}

Cada subruta ten, ademáis, unha subruta \texttt{/clear} que permite borrar a caché facéndolle un POST, opcionalmente engadindo a petición a borrar.

\section{Organización do código}

\subsection{Servidor}

A definición das API atópase dividida en tres clases: UsersResource para \texttt{/users}, TemplateResource para \texttt{/templates} e ComponentResource para \texttt{/components}. Dentro desas clases defínese un método por ruta. A clase OpenApiResource indica baixo que ruta pode accederse á definición OpenAPI.

\newpage

No paquete raíz atópanse tamén unha serie de clases auxiliares ou de configuración:

\begin{itemize}
	\item O punto de entrada da aplicación en Application
	\item Unha serie de valores predeterminados na base de datos, en DataLoader
	\item A configuración de seguridade, en AuthenticationManager. Véxase o capítulo \ref{chap:seguridade}
	\item A definición dunha cadea de texto serializable por JSON. Usando a clase String directamente Jakarta entende que o método devolve JSON xa serializado, e causa problemas ao enviar e recibir cadeas.
\end{itemize}

O paquete \texttt{exceptions} recolle todas as excepcións, separadas en subpaquetes. Tamén inclúe unha excepción xenérica, \texttt{RESTException}, que é superclase de todas as excepcións. O serializador \texttt{RESTExceptionMapper} encárgase de serializar os \texttt{RESTException}, transformando a un \texttt{RESTExceptionSerializable} só cos atributos que se queren enviar. Doutro xeito, se se enviase directamente o \texttt{RESTException} este incluiría tamén atributos da clase \texttt{Exception}, que non interesan; e a clase \texttt{RESTExceptionSerializable} non se pode lanzar coma excepción ao non ser subclase de \texttt{Exception}.

O paquete \texttt{dao} recolle as definicións dos datos, e tamén as interfaces que definen as posibles peticións ás bases de datos. Hibernate encárgase de xerar peticións usando os nomes dos métodos definidos nas interfaces.\cite{jpaquery}

Por mor dun fallo de OpenAPI\cite{jsondate} non se usan directamente as clases de tempo de Java, polo que no seu lugar úsase unha versión propia \texttt{JSONDatetime}, que inclúe métodos \textit{getter}, \textit{setter} e \texttt{parse} (ademais do construtor); e encapsula un \texttt{OffsetDatetime}, que é a clase que usa OpenAPI para datos de tipo temporal.

\subsection{Cliente}

O cliente organízase de xeito semellante ao servidor. No paquete raíz atópanse o punto de entrada (Main), a configuración de SecurityConfig (véxase o capítulo \ref{chap:seguridade}). Tamén ten paquetes para as excepcións, para a seguridade e un chisco de configuración en \texttt{service} e para as vistas. A clase CSSandJSResources encárgase de procesar as peticións ás rutas \texttt{/css/} e \texttt{/js/} e devolver os recursos axeitados.

\section{Busca}
\label{sec:busca}

O sistema de busca está composto de dúas fases. Na primeira, convértese a cadea da petición nunha árbore de nodos usando ANTLR. Despois, e de xeito recursivo, compróbase obxecto a obxecto se se cumplen as condicións.

A definición usada por ANTLR para xerar o código axeitado atópase no ficheiro Find.g4, na ruta \texttt{pleste-server\allowbreak{}/src\allowbreak{}/main\allowbreak{}/antrl4\allowbreak{}/gal\allowbreak{}/udc\allowbreak{}/fic\allowbreak{}/prperez\allowbreak{}/pleste\allowbreak{}/service}. As diferenzas máis importantes entre a gramática Backus–Naur indicada no capítulo \ref{chap:deseno} e o implementado no ficheiro son engadir a regra «top», que contén só unha \texttt{<expresión>} e o carácter EOF por mor dun fallo en ANTLR que non se vai arranxar\cite{antlr118}; e unha definición un chisco distinta de cadeas, para que o analizador léxico non consuma o carácter previo ás comiñas de inicio.

Tras executar a definición sobre unha cadea de entrada (véxase a sección \ref{sec:tec_servidor}, no parágrafo referente a ANTLR) xérase unha árbore propia, formada por unha serie de nodos, cada ún dos cales implementa un método \texttt{matches(Object)}, que comproba se o obxecto indicado cumpre coa condición indicada (no caso da regra \texttt{<proba>}) ou devolve o resultado de chamar, recursivamente, este método nos seus nodos fillos (no caso das regras \texttt{<expresión>}).

A estrutura destes nodos é a seguinte:

\begin{itemize}
	\item Node: Clase abstracta contén un atributo \texttt{child} que conten o único fillo deste nodo
	
	\begin{itemize}
		\item RootNode: Versión concreta da clase Node, usada só para o nodo raíz da árbore.
		\item TestIn: Comproba sobre cadeas
		\item TestNumber: Comproba sobre comparacións numéricas
		\item TestString: Comproba sobre unha cadea
		\item BinaryNode: Esta clase abstracta introduce un segundo nodo fillo, para poder implementar unha árbore binaria.
		
		\begin{itemize}
			\item AndNode: Implementa a operación de conxunción sobre os dous nodos fillos
			\item OrNode: Implementa a operación de disxunción sobre os dous nodos fillos
		\end{itemize}
	\end{itemize}
\end{itemize}

\newpage

Deste xeito, a árbore da figura \ref{fig:arboresintaxe} quedaría tal como se amosa na figura \ref{fig:arborenodos}:

\begin{figure}[H]
	\centering
	\resizebox{.8\linewidth}{!}{\begin{tikzpicture}
\Tree [.RootNode
	[.OrNode
		[.OrNode
			TestIn
			TestIn
		]
		[.OrNode
			TestString
			[.OrNode
				TestIn
				TestNumber
			]
		]
	]
]
\end{tikzpicture}}
	\caption{Árbore de sintaxe de exemplo}
	\label{fig:arborenodos}
\end{figure}

Decidiuse usar un método recursivo, e non iterativo, por que resultaba máis sinxelo implementalo e non se considera que as posibles peticións que se fixeran poidan sobrepasar o límite da pila de chamadas. De todos xeitos, o uso dos operadores lóxicos de Java (\texttt{\&\&} e \texttt{||}) con cortocircuíto reducen tamén o número de operacións a realizar.

Para implementar a paxinación úsase unha caché LRU, que usando coma índice do mapa a petición devolve unha listaxe de obxectos comprobados para esa petición. Existe unha caché para cada tipo de obxecto.

As propiedades dispoñibles son:

\begin{itemize}
	\item template.name
	\item template.description
	\item template.field.name
	\item template.field.type
	\item component.name
	\item component.description
	\item component.field.name
	\item component.field.type
	\item component.field.value
	\item user.name
	\item user.email
	\item user.role
\end{itemize}

Comparar un valor dun campo só devolverá os compoñentes con campos do tipo axeitado (\texttt{TEXT}, \texttt{DATETIME} para as cadeas; \texttt{NUMBER}, \texttt{LINK} e \texttt{DATETIME} para os números). Para comparar con propiedades que conteñan enumeracións, deberá usarse o nome orixinal da enumeración, non o que poida amosar a interface web.